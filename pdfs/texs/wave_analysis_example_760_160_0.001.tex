\documentclass[12pt,a4paper]{article}
\usepackage[utf8]{inputenc}
\usepackage[russian]{babel}
\usepackage{graphicx}
\usepackage{float}
\usepackage{geometry}
\geometry{margin=2cm}
\usepackage{booktabs}
\usepackage{longtable}

\begin{document}

\section*{Анализ затухания волн: L=760, DW=160, R=0.001}

\subsection*{Параметры моделирования}

\begin{itemize}
    \item Длина образца: $L = 760$ м
    \item Ширина образца: $DW = 160$ м
    \item Коэффициент затухания: $R = 0.001$
    \item Модуль Юнга: $E$, Коэффициент Пуассона: $\nu$, Плотность: $\rho$
\end{itemize}

\subsection*{Расчётные скорости волн}

\begin{itemize}
    \item Аналитическая скорость продольной волны: $v_p = 376.43$ м/с
    \item Аналитическая скорость поперечной волны: $v_s = 201.21$ м/с
    \item Аналитическая скорость Рэлеевской волны: $v_r = 186.35$ м/с
    \item Временной шаг: $\Delta t = 3.41e-04$ с
    \item Экспериментальная скорость поперечной волны: $v_s^{exp} = 250.00$ м/с
    \item Экспериментальная скорость отраженной продольной волны: $v_p^{ref} = 192.25$ м/с
    \item Величина ошибки: $\varepsilon = 1.01e-07$
\end{itemize}

\subsection*{Анализ прихода продольной волны}

\begin{itemize}
    \item Аналитическое время прихода продольной волны: $t_{analytical} = 2.02$ с
    \item Экспериментальный индекс прихода продольной волны: 6047
    \item Экспериментальное время прихода продольной волны: $t_{wave} = 2.06$ с
    \item Скорость пришедшей продольной волны: $v_p^{exp} = 368.80$ м/с
    \item Длина продольной волны в индексах: 1359
    \item Длина продольной волны в метрах: $\lambda_p = 170.77$ м
\end{itemize}

\subsection*{Анализ пересечения волн}

\begin{itemize}
    \item Экспериментальное время пересечения волн: $t_{intersection} = 2.61$ с
    \item Экспериментальный индекс пересечения волн: 7672
    \item Экспериментальное расстояние пересечения волн: $x_{intersection} = 653.57$ м
    \item Номер приемника, ближайшего к пересечению: 33
    \item Индекс прихода падающей волны через ресивер пересечения: 5251
\end{itemize}

\subsection*{Визуализация тензора}

\begin{figure}[H]
    \centering
    \includegraphics[width=0.95\textwidth]{./pdfs/figures/wave_analysis_example_760_160_0.001_tensor_imshow.png}
    \caption{Визуализация распространения волн в среде}
    \label{fig:tensor}
\end{figure}

\subsection*{Анализ волн}

\begin{figure}[H]
    \centering
    \includegraphics[width=0.95\textwidth]{./pdfs/figures/wave_analysis_example_760_160_0.001_waves_analysis.png}
    \caption{Анализ падающих и отраженных волн}
    \label{fig:waves}
\end{figure}

\subsection*{Результаты анализа амплитуд}

\begin{longtable}{|c|c|c|c|c|c|c|c|c|c|}
\hline
Позиция & $t_{p,fall}$ & $t_{p,refl}$ & $t_{s,fall}$ & $t_{int}$ & $A_{p,fall}$ & $A_{p,refl,0}$ & $A_{p,refl,1}$ & Ratio 0 & Ratio 1 \\
(м) & (с) & (с) & (с) & (с) & & & & (\%) & (\%) \\
\hline
\endfirsthead
\hline
Позиция & $t_{p,fall}$ & $t_{p,refl}$ & $t_{s,fall}$ & $t_{int}$ & $A_{p,fall}$ & $A_{p,refl,0}$ & $A_{p,refl,1}$ & Ratio 0 & Ratio 1 \\
\hline
\endhead
\hline
\endfoot
720.0 & 1.952 & 2.269 & 2.880 & 2.614 & 4.610e-05 & 1.783e-05 & 2.954e-05 & 38.68\% & 64.09\% \\

\hline
\end{longtable}

\end{document}
